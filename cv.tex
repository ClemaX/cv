%% start of file `cv.tex'.
\documentclass[11pt, a4paper, sans]{moderncv}

\moderncvtheme[blue]{classic}

\usepackage[french]{babel}
\frenchsetup{StandardItemLabels=true}

\usepackage[T1]{fontenc}
\usepackage[utf8]{inputenc}
\usepackage{lmodern}
\usepackage{microtype}

\usepackage[scale=0.8]{geometry}

\usepackage{ClementHamada}

\title{Architecte du numérique}

\quote{\foreignlanguage{english}{Do \textbf{one thing}, and do it \textbf{well}. - UNIX}}

\begin{document}

\makecvtitle

\section{Compétences}
\cvlistitem{Très bonne capacité de communication. Trilingue français, anglais et allemand.}
\cvlistitem{Bonne capacité à travailler en groupe ainsi qu'en autonomie.}
\cvlistitem{Extrêmement motivé pour développer de nouvelles compétences.}
\cvlistitem{Mon entourage me considère comme quelqu'un de sociable, sérieux et créatif.}

\subsection{Développement}
\cvdoubleitem{Langages}{C, C++, Python, Shell, Make}{Web}{HTML, CSS, JS, WS, TypeScript}
\cvdoubleitem{Frameworks}{NestJS, ReactJS, Flutter}{BDD}{TypeORM, MySQL, PostgreSQL}

\subsection{Administration}
\cvdoubleitem{Web}{Nginx, VsFTPd, Graphana}{Sécurité}{SSL, HTTPS, SSH, PGP/GPG}
\cvdoubleitem{Systèmes d'exploitation}{Linux (Debian, Ubuntu, Arch),\\macOS, Windows}{Virtualisation}{Docker, Compose, Kubernetes,\\VirtualBox, Amazon AWS}


\section{Éducation}
\cventry{2019--2022}{Architecte du numérique}{42 Lyon}{France}{}{}
\cventry{2014--2019}{Baccalauréat}{\foreignlanguage{ngerman}{Gymnasium Spaichingen}}{Allemagne}{}{}
\cventry{2011--2014}{Brevet}{Collège de Groisy}{France}{}{}


\section{Expérience}
\cventry{2012}{Découverte de la programmation}{Le site du zéro}{}{}{C++/Java}
\cventry{2013}{Montage de mon premier PC}{Tour ATX}{}{}{}
\cventry{2014}{Découverte de Linux et Hackintosh}{}{}{}{Raspberry Pi, Serveur WordPress, Scripts shell}
\cventry{2015}{Pilotage d'un ruban LED RGB par HTTP}{}{}{}{Développement d'un backend et d'interfaces web et infrarouge sur Raspberry Pi}
\cventry{2016}{Stage de 3e}{i-mation}{Allemagne}{}{Automatisation de chaînes de production grâce à la vision par ordinateur}
\cventry{2017}{Montage d'une imprimante 3D}{RepRap/i3}{}{}{Slicing G-Code, programmation d'objets 3D paramétriques avec OpenSCAD}
\cventry{2018}{Découverte du développement d'applications mobiles}{Android, IOS}{}{}{Android Studio, Swift, Flutter}

\end{document}
