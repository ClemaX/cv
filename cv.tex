%% start of file `cv.tex'.

\documentclass[11pt, a4paper, sans]{moderncv}

\moderncvtheme[blue]{classic}

\usepackage[scale=0.75]{geometry}

\usepackage{ClementHamada}

\begin{document}

\makecvtitle

\section{Compétences}
\cvlistitem{Très bonne capacité de communication. Trilingue francais, anglais et allemand.}
\cvlistitem{Bonne capacité a travailler en groupe ainsi qu'en autonomie.}
\cvlistitem{Extremement motivé pour développer de nouvelles competences.}
\cvlistitem{Mon entourage me considère comme quelqu'un de sociable, serieux et éfficace.}

\section{Education}
\cventry{2019--2022}{Architecte du numérique}{42 Lyon}{France}{}{}
\cventry{2014--2019}{Baccalauréat}{Gymnasium Spaichingen}{Allemagne}{}{}
\cventry{2011--2014}{Collège}{Collège de Groisy}{France}{}{}

\section{Experience}
\cventry{2012}{Découverte de la programmation}{Le site du zero}{}{}{C++/Java}
\cventry{2013}{Montage de mon premier PC}{Tour ATX}{}{}{}
\cventry{2014}{Découverte de Linux et Hackintosh}{}{}{}{Raspberry Pi, Serveur WordPress, Scripts shell}
\cventry{2015}{Pilotage d'un ruban LED RGB par HTTP}{}{}{}{Développement d'un backend et d'interfaces web et infrarouge sur Raspberry Pi}
\cventry{2016}{Stage de 3ème}{i-mation}{Allemagne}{}{Automatisation de chaînes de production grace a la vision par ordinateur}
\cventry{2017}{Montage d'une imprimante 3D}{RepRap/i3}{}{}{Slicing G-Code, programmation d'objets 3D paramétriques avec OpenSCAD}
\cventry{2018}{Découverte du développement d'applications mobiles}{Android, IOS}{}{}{Android Studio, Swift, Flutter}

\end{document}