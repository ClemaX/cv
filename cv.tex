%% start of file `cv.tex'.
\documentclass[11pt, a4paper, sans]{moderncv}

% Theme
\moderncvtheme[blue]{classic}

\usepackage[english,ngerman,french]{babel}
\frenchsetup{StandardItemLabels=true}

% Encoding
\usepackage[T1]{fontenc}
\usepackage[utf8]{inputenc}
\usepackage{lmodern}
\usepackage{microtype}

% Geometry
\usepackage[scale=0.8,vscale=0.85]{geometry}
\settowidth{\hintscolumnwidth}{Décembre 2020}
\recomputelengths

% Identity
\usepackage{ClementHamada}

\title{Architecte du numérique}

\quote{\foreignlanguage{english}{Do \textbf{one thing}, and do it \textbf{well}. - UNIX}}

\begin{document}

\makecvtitle

\section{Compétences}
\cvlistitem{Très bonne capacité de communication. Trilingue français, anglais et allemand.}
\cvlistitem{Capacité à travailler en groupe ainsi qu'en autonomie.}
\cvlistitem{Extrêmement motivé pour développer de nouvelles compétences.}
\cvlistitem{Sociable, sérieux et créatif.}

\subsection{Développement}
\cvcomputer{Langages}{C, C++, Python, Shell, Make}{Web}{HTML, CSS, JS, OAuth, TypeScript}
\cvcomputer{Frameworks}{NestJS, React, TypeORM, Flutter}{BDD}{MySQL, PostgreSQL, InfluxDB}
\cvcomputer{Méthodes}{POO, MVC, multithreading, unit-test}{Outils}{Node.js, Git, GitHub, GDB/LLDB}

\subsection{Administration}
\cvcomputer{Web}{Nginx, VsFTPd, Graphana}{Sécurité}{SSL, HTTPS, SSH, PGP/GPG}
\cvcomputer{Systèmes d'exploitation}{Linux (Debian, Ubuntu, Arch),\\macOS, Windows}{Virtualisation}{Docker, Compose, Kubernetes,\\VirtualBox, Amazon AWS}


\section{Formations}
\cventry{2019--2022}{Architecte du numérique}{École 42}{Lyon(69)}{France}{}
\cventry{2019}{Baccalauréat}{}{\foreignlanguage{ngerman}{Spaichingen}}{Allemagne}{}
\cventry{2014}{Brevet des collèges}{}{Groisy(74)}{France}{}


\section{Expériences}
%\cventry{Octobre 2021}{\foreignlanguage{english}{snow-crash}}{Introduction a la sécurité informatique}{}{}{Exploitation d'exécutables, ingénierie inverse}
\cventry{Aout 2021}{Transcendance}{Jeu multijoueur dans le navigateur web}{}{}{Node.js, React, NestJS, PostgreSQL, TypeORM, Socket.io, OAuth}
\cventry{Juin 2021}{Serveur IRC}{Serveur de messagerie respectant le protocole IRC}{}{}{C++, Socket}
\cventry{Mars 2021}{Piscine C++}{Exercices visant l'apprentissage du langage C++}{}{}{Programmation orientée objets}
\cventry{Février 2021}{\foreignlanguage{english}{Philosophers}}{Problème de partage de ressource en \foreignlanguage{english}{multithreading}}{}{}{C, pthread}
\cventry{Novembre 2020}{Services}{Orchestration d'un cluster virtuel composé de plusieurs services}{}{}{Kubernetes, Docker, WordPress, phpMyAdmin, FTP, Graphana, Nginx, reverse-proxy}
\cventry{Février 2020}{Cub3D}{Moteur de jeu utilisant le \foreignlanguage{english}{ray-casting}}{}{}{C, minilibx}
\cventry{Aout 2019}{Piscine C}{Épreuve d'un mois visant l'apprentissage du langage C}{}{}{}



% \cventry{2012}{Découverte de la programmation}{Le site du zéro}{}{}{C++/Java}
% \cventry{2013}{Montage de mon premier PC}{Tour ATX}{}{}{}
% \cventry{2014}{Découverte de Linux et Hackintosh}{}{}{}{Raspberry Pi, Serveur WordPress, Scripts shell}
% \cventry{2015}{Pilotage d'un ruban LED RGB par HTTP}{}{}{}{Développement d'un backend et d'interfaces web et infrarouge sur Raspberry Pi}
% \cventry{2016}{Stage de 3e}{i-mation}{Allemagne}{}{Automatisation de chaînes de production grâce à la vision par ordinateur}
% \cventry{2017}{Montage d'une imprimante 3D}{RepRap/i3}{}{}{Slicing G-Code, programmation d'objets 3D paramétriques avec OpenSCAD}
% \cventry{2018}{Découverte du développement d'applications mobiles}{Android, IOS}{}{}{Android Studio, Swift, Flutter}

\end{document}
